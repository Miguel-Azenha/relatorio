

Este capítulo apresenta os fundamentos teóricos necessários para compreender o desenvolvimento 
de aplicações baseadas em \textit{Trusted Execution Environments} (TEEs), com particular 
foco na tecnologia Intel Software Guard Extensions (SGX), utilizada neste trabalho para 
proteger dados sensíveis de uma carteira eletrónica. São explorados os conceitos fundamentais 
de enclaves, modelo de ameaça, comunicação segura entre domínios de confiança e mecanismos 
de selagem (\textit{sealing}) para armazenamento persistente.

\subsection{Trusted Execution Environments}

Um \textit{Trusted Execution Environment} (TEE) é um ambiente seguro, isolado do sistema 
operativo e das aplicações comuns, que se destina à execução de código sensível. O objetivo 
dos TEEs é proteger dados em uso, enfrentando uma limitação de segurança tradicional, embora a 
criptografia proteja dados em repouso e em trânsito, estes permanecem vulneráveis quando 
estão a ser processados na memória de sistemas potencialmente comprometidos.

Os TEEs garantem:
\begin{itemize}
    \item \textbf{Confidencialidade}: dados dentro do TEE não podem ser acedidos por software externo
    \item \textbf{Integridade}: o código e dados protegidos não podem ser modificados
    \item \textbf{Isolamento}: o estado do TEE é independente do sistema operativo 
    \item \textbf{Autenticidade}: o TEE permite provar a sua identidade e integridade 
\end{itemize}

SGX e TrustZone representam abordagens distintas dentro do mesmo paradigma de computação 
confiável, o SGX fornece enclaves ao nível de aplicação, enquanto TrustZone divide todo o 
processador em duas áreas, uma segura e outra não segura.

\subsection{Intel Software Guard Extensions}

Intel SGX é uma tecnologia de isolamento de memória que permite criar enclaves, regiões 
protegidas que executam código de forma isolada do sistema operativo, BIOS, firmware, 
hipervisor ou aplicações externas. O mecanismo baseia-se na Enclave Page Cache (EPC), uma 
área de memória física protegida por hardware e inacessível a processos externos, mesmo com 
privilégios elevados.

\subsection{Arquitetura dos Enclaves}

Um enclave é uma biblioteca compilada como \textit{shared object} contendo:
\begin{itemize}
    \item \textbf{Código confiável};
    \item \textbf{Dados sensíveis};
    \item \textbf{Metadados de segurança};
    \item \textbf{Tamanho do heap}, \textbf{stack} e número de \textbf{trusted threads}.
\end{itemize}

Estes metadados são utilizados pelo carregador não confiável para preparar a EPC, mas sem 
nunca aceder ao conteúdo real do enclave.

De acordo com a Intel, assumem-se como não confiáveis o sistema operativo, hipervisor, drivers, firware e aplicações externas.
O atacante pode controlar todo o sistema operativo e observar a memória, mas não consegue let ou alterar os dados dentro do enclave, pois eles estão protegidos por hardware.


\section{Interface entre Código Confiável e Não Confiável}

A comunicação entre a aplicação não confiável e o enclave é feita com base em:
\begin{description}
    \item[ECALLs] chamadas da aplicação para o enclave (entrada);
    \item[OCALLs] chamadas do enclave para a aplicação (saída).
\end{description}

Esta interface é descrita no ficheiro \texttt{.edl}, que especifica:
\begin{itemize}
    \item tipos de dados permitidos atravessar a fronteira de confiança;
    \item tamanhos de buffers;
    \item direções de ponteiros;
    \item validação automática feita pelo proxy gerado pelo \texttt{sgx\_edger8r}.
\end{itemize}


\section{Selagem e Armazenamento Seguro}

A selagem (\textit{sealing}) é um mecanismo essencial para permitir persistência de dados 
confidenciais. Embora o enclave ofereça proteção em memória, os seus dados são perdidos 
quando o mesmo é encerrado. Assim, é necessário guardar estes dados num ficheiro ou base de 
dados externa, mas de forma criptograficamente protegida antes de os dados deixarem o 
enclave.


\section{Minimização da Trusted Computing Base}

Um dos princípios fundamentais na construção de TEEs é a minimização da TCB — o conjunto de 
código confiável cuja segurança é crítica. Tal como discutido nas aulas, um enclave deve 
conter:
\begin{itemize}
    \item apenas a lógica que manipula dados sensíveis;
    \item apenas funções essenciais;
    \item estruturas de dados estritamente necessárias.
\end{itemize}

No projeto:
\begin{itemize}
    \item toda a criptografia, verificação de passwords, manipulação da carteira e geração 
          segura de passwords ocorrem no enclave;
    \item a interface é reduzida a um conjunto mínimo de ECALLs.
\end{itemize}


\section{Síntese}

O enquadramento teórico apresentado demonstra que Intel SGX fornece um TEE adequado para 
proteção de dados sensíveis ao nível da aplicação, oferecendo isolamento forte e suporte 
nativo para selagem. Estas capacidades tornam SGX ideal para a construção de uma carteira 
segura persistente, tal como implementado neste trabalho.
